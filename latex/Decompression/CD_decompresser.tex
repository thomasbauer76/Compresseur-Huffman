\begin{algorithme}
	\fonction{decompresser}{fCompresse : \fichierbinaire}{\fichierbinaire}{}{fDecompresse : \fichierbinaire, id, longueur : \naturel, s : \statistiques, abh : \arbredehuffman}
    {
    	\sialors{estOuvert(f)}
        {
            \instruction{fermer(f)}
        }
        \instruction{ouvrir(f, lecture)}
        \\
        \commentaire{Les instructions ci-dessus correspondent à une réinitialisation du curseur en tête de fichier.}
    	\\
    	\commentaire{Lecture de l'identifiant (un \textit{unsigned short int})}
    	\instruction{lireNaturel(fCompresser, id)}
    	\\
    	\commentaire{On vérifie si on retrouve bien notre identifiant (ici 1000 sous sa forme de naturel), si ce n'est pas le cas, on n'a pas a décompresser le fichier et l'on retourne fCompresser en tant que fichier décompressé}
    	\sialors{id != 1000}{
    		\instruction{retourner fCompresser}
    	}
		\commentaire{Lecture de la longueur (un \textit{unsigned long long int})}    
    	\instruction{lireNaturel(fCompresser, longueur)}
    	\instruction{lireStatistiques(fCompresser, s)}
    	\instruction{construireArbreDeHuffman(s, abh)}
    	\instruction{decoder(abh, fCompresse, fDecompresse)}
    	\\
    	\commentaire{Normalement on est censé vérifier que la longueur de FDecompresse est égale à la longueur du fichier originel (stocké dans la variable longueur) mais je ne sais pas comment écrire ça en pseudo-code}
    	\instruction{libererArbreDeHuffman(abh)}
    	\\
    	\commentaire{L'instruction au dessus permet de s'assurer de la libération de l'ArbreDeHuffman (conseillé par le prof)}
    }
\end{algorithme}