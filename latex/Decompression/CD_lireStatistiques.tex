\begin{algorithme}
    \fonction{lireStatistiques}{fb : \fichierbinaire}{\statistiques}{}
    {octetInutile, o : \octet , s : \statistiques, i : \naturel}
    {
		\sialors{non((estOuvert(fb)) et (mode(fb) $=$ lecture))}{    	
    		\instruction{ouvrir(fb, lecture)}
    	}
    	\\
    	\textcolor{gray}{//La prochaine boucle a pour objectif de déplacer le curseur à l'endroit au se situe les statistiques, c'est-à-dire après l'identifiant(d'une taille de 2 octets) et après la longueur (de la taille d'un \textit{unsigned long int}. Cependant, il existe sûrement une fonction en C qui permet de seulement déplacer le curseur sans sortir l'octet sur lequel on vient de passer (contrairement aux opérations du TAD fourni dans le sujet)}
    	\pour{i}{1}{(2*taille(\textit{octet}) + taille(\textit{unsigned long int})) / taille(\textit{octet})}{}
    	{
    		\instruction{lireOctet(fb, octetInutile)}
    	}
		\\
    	\textcolor{gray}{//Maintenant le curseur est au bon endroit et l'on peut lire les statistiques}
    	\pour{i}{0}{255}{}
    	{
    		\instruction{lireOctet(fb, o)}
        	\affecter{s[i]}{octetVersNaturel(o)}
    	}
    }
\end{algorithme}