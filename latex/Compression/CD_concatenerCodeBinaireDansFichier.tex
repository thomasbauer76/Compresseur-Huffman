\begin{algorithme}
     \procedure{concatenerCodeBinaireDansFichier}{\paramEntreeSortie{cbtemp : \codebinaire},\paramEntree{cb : \codebinaire},\paramEntreeSortie{fbCompresse : \fichierbinaire}}
    {}
    { i,j, taillecbtemp, taillecb : Entier, o : \octet}
    {
        		\affecter{taillecbtemp}{obtenirLongeur(cbtemp)}
        		\affecter{taillecb}{obtenirLongeur(cb)}
        		\sialors{taillecbtemp=8}{\affecter{cbtemp}{creeCodeBinaire(obtenirIemeBit(cb,0)}
        		\sialorssinon{taillecbtemp+taillecb $>$ 8}
        		{		
        			\pour{i}{1}{8 $-$ taillecb}{}
        			{
        					\instruction{ajouterBit(cbtemp,obtenirIemeBit(cb,i)}
					}
					\affecter{o}{codebinaireEnOctet(cb)}       											\instruction{ecrireOctet(fbCompresse,o)}
        			\affecter{cbtemp}{creeCodeBinaire(obtenirIemeBit(cb,i$+$1)}
        			\pour{j}{i$+$1}{taillecb}{}
        			{
        					\instruction{ajouterBit(cbtemp,obtenirIemeBit(cb,i)}
        			}
        		}
        		{		\pour{i}{taillecbtemp $+$ 1}{ taillecbtemp$+$ taillecb}{}
        			{
        					\instruction{ajouterBit(cbtemp,obtenirIemeBit(cb,i)}
        			}
        			\affecter{taillecbtemp}{taillecbtemp $+$ taillecb}
        			\sialors{taillecbtemp $=$ 8}{
        				\affecter{o}{codebinaireEnOctet(cb)}       											\instruction{ecrireOctet(fbCompresse,o)}
        			}	
        		}
    }
    }
\end{algorithme}
    	    			  			
    	    			  						   