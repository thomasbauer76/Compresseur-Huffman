\begin{algorithme}
     \procedure{concatenerCodeBinaireDansFichier}{\paramEntreeSortie{cbtemp : \codebinaire},\paramEntree{cb : \codebinaire},\paramEntreeSortie{fbCompresse : \fichierbinaire}}
    {}
    { i,j, tailleCbTemp, taillecb : Entier, o : \octet}
    {
        		\sialors{obtenirLongeur(cbtemp) = 8}
				{											
					\affecter{cbtemp}{cb}
				}
				\affecter{tailleCbTemp}{obtenirLongeur(cbtemp)}
        		\affecter{tailleCb}{obtenirLongeur(cb)}
				\affecter{tailleTotale}{tailleCbTemp + tailleCb}
        		\sialorssinon{tailleTotale $>$ 8}
        		{		
        			\pour{i}{0}{7 $-$ tailleCbTemp}{}
        			{
        				\instruction{ajouterBit(cbtemp,obtenirIemeBit(cb,i))}
					}    											
					\instruction{ecrireOctet(fbCompresse, codeBinaireEnOctet(cb))}
					\\
        			\affecter{cbtemp}{creeCodeBinaire(obtenirIemeBit(cb,i$+$1))}
        			\pour{j}{ i $+$ 2}{tailleCb $-$ 1}{}
        			{
        					\instruction{ajouterBit(cbtemp,obtenirIemeBit(cb,j))}
        			}
        		}
        		{		
					\pour{i}{tailleCbTemp}{tailleTotale $-$ 1}{}
        			{
        					\instruction{ajouterBit(cbtemp,obtenirIemeBit(cb,i))}
        			}
					\sialors{obtenirLongueur(cbTemp) $=$ 8}
					{
						\instruction{ecrireOctet(fbCompresse, codeBinaireEnOctet(cbtemp))}
					}
        		}
    }
\end{algorithme}
    	    			  			
    	    			  						   