\begin{algorithme}
     \procedure{concatenerCodeBinaireDansFichier}{\paramEntreeSortie{cbTemp : \codebinaire},\paramEntree{cb : \codebinaire},\paramEntreeSortie{fbCompresse : \fichierbinaire}}
    {}
    {i, j, tailleCb, tailleLibre : \naturel}
    {
		\affecter{tailleCb}{obtenirLongeur(cb)}
		\affecter{tailleLibre}{8 - obtenirLongueur(cbTemp)}	
		\\
		\sialorssinon{obtenirLongueur(cbTemp) = 8}
		{											
			\affecter{cbTemp}{cb}
			\instruction{\commentaire{On assigne tailleCb + 1 à i pour ne pas entrer dans la dernière condition de la procédure}}
			\affecter{i}{tailleCb + 1}
		}
		{
			\pour{i}{ 0}{min(tailleCb, tailleLibre) - 1}{}
			{
				\instruction{ajouterBit(cbTemp, obtenirIemeBit(cb, i))}
			}  
		}
		\\
		\sialors{obtenirLongueur(cbTemp) = 8}
		{											
			\instruction{ecrireOctet(fbCompresse, codeBinaireEnOctet(cb))}
		}
		\\
		\sialors{i < tailleCb - 1}
		{
			\affecter{cbTemp}{creeCodeBinaire(obtenirIemeBit(cb, i $+$ 1))}
			\pour{j}{ i $+$ 2}{tailleCb $-$ 1}{}
			{
					\instruction{ajouterBit(cbTemp, obtenirIemeBit(cb, j))}
			}
		}
    }
\end{algorithme}
    	    			  			
    	    			  						   