\begin{algorithme}
    \procedure{obtenirTableDeCodageRecursif}{\paramEntreeSortie{tdc : \tabledecodage}, \paramEntree{a : \arbredehuffman, cb : \codebinaire}}
    {}
    {cbCopie : \codebinaire}
    {
        \sialorssinon{estUneFeuille(a)}
        {
            \instruction{ajouterCodage(tdc, obtenirOctet(a), cb)}
        }
        {
            \affecter{cbCopie}{cb}
            \instruction{ajouterBit(cbCopie, bitA0)}
            \instruction{obtenirTableDeCodageRecursif(tdc, obtenirFilsGauche(a), cbCopie)}
            \instruction{ajouterBit(cb, bitA1)}
            \instruction{obtenirTableDeCodageRecursif(tdc, obtenirFilsDroit(a), cb)}
        }
    }
    
    \fonction{obtenirTableDeCodage}{a : \arbredehuffman}{\tabledecodage}
    {}
    {tdc : \tabledecodage, cbGauche, cbDroit : \codebinaire}
    {
        \affecter{tdc}{creerTableDeCodage()}
        \\
        \commentaire{Si a est une feuille, alors le fichier ne contient qu'un octet (qui peut-être présent plusieurs fois).}
        \\
        \commentaire{L'arbre n'a donc pas de fils. On encode alors le seul octet avec le code binaire bitA0.}
        \sialorssinon{estUneFeuille(a)}
        {
            \instruction{ajouterCodage(tdc, obtenirOctet(a), creerCodeBinaire(bitA0))}
        }
        {
            \affecter{cbGauche}{creerCodeBinaire(bitA0)}
            \affecter{cbDroit}{creerCodeBinaire(bitA1)}
            \instruction{obtenirTableDeCodageRecursif(tdc, obtenirFilsGauche(a), cbGauche)}
            \instruction{obtenirTableDeCodageRecursif(tdc, obtenirFilsDroit(a), cbDroit)}
            \retourner{tdc}
        }
    }
\end{algorithme}