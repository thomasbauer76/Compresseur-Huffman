\begin{algorithme}
 \procedure{encoder}{\paramEntreeSortie{fbInitial : \fichierbinaire},\paramEntree{ tdc : \tabledecodage},\paramSortie{fbCompresse : \fichierbinaire}}
    {}
    {cbtemp, cb: \codebinaire, o : \octet, i,j, taillecb, taillecbtemp : Entier}
    {
        
        
        \sialors{estOuvert(fbInitial)}
        {
            \instruction{fermer(fbInitial)}
        }
        \instruction{ouvrir(fbInitial,lecture)}
        \sialors{estOuvert(fbCompresse)}
        {
            \instruction{fermer(fbCompresse)}
        }
        \instruction{ouvrir(fbCompresse, ecriture)}
        
        \instruction{ecrirIdentifiant(fbCompresse)}
        \instruction{ecrireTailleFichier(fbCompresse,obtenirTailleDuFichier(fbInitial))}
  %% le premier octet represente la taille du fichier initale à utiliser ou non 
        \instruction{ecrireStatistique(fbCompresse,s)}
         {
        \affecter{cbtemp}{creerCodeBinaire(bitA0)}
        \pour{i}{1}{7}{}{\instruction{ajouterBit(restecbtemp,bitA0)}}
        }
        \tantque{$non$ finfichier(fnInitiale)}{
       		
        		\instruction{lireoctet(fbInitiale,o)}
        		\affecter{cb}{octetVersCodeBinaire(tdc,o)}
        	 	\instruction{concatenerCodeBinaireDansFichier(cbtemp,cb,fbcompresse)}
        		}
        
        \sialors{obtenirLongueur(cbTemp) $<$ 8}
		{
			\pour{i}{tailleCbTemp}{tailleTotale - 1}{}
        			{
        					\instruction{ajouterBit(cbtemp,bitA0)}
        			}
        	\instruction{ecrireOctet(fbCompresse, codeBinaireEnOctet(cbtemp))}
			}
			
       	 	\instruction{fermer(fbInitial)}
        	\instruction{fermer(fbCompresse)}
        	
    }
\end{algorithme}
    	    	