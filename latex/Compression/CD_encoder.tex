\begin{algorithme}
     \procedure{encoder}{\paramEntreeSortie{fbInitial : \fichierbinaire},\paramEntree{ tdc : \tabledecodage},\paramSortie{fbCompresse : \fichierbinaire}}
    {}
    {cbtemp, cb: \codebinaire, oInitial, oCompresse : \octet, i,j, taillecb, taillecbtemp : Entier}
    {
        \affecter{fbInitial}{fichierBinaire("nom du fichier")}
        \instruction{ouvrir(fbInitial,lecture)}
        \affecter{fbCompresse}{fichierBinaire("nom du fichier")}
        \instruction{ouvrir(fbCompresse, ecriture)}
        \instruction{ecrireoctet(naturelVersOctet(obtenirTailleFichier(fbInitiale)),fbcompresse)}
  %% le premier octet represente la taille du fichier initale à utiliser ou non 
        \instruction{ecrireoctet(creeoctet(0,0,0,0,0,0,1,1)}
        \instruction{ecrireoctet(creeoctet(1,1,1,0,1,0,0,0)}
        
        \instruction{ecrireStatistique()}
         {
        \affecter{cbtemp}{creerCodeBinaire(0)}
        \pour{i}{1}{7}{}{\instruction{ajouterBit(restecbtemp,0)}}
        }
        \tantque{$non$ finfichier(fnInitiale)}{
       		
        		\instruction{lireoctet(fbInitiale,oInitial)}
        		\affecter{cb}{octetVersCodeBinaire(tdc,oInitial)}
        	  \instruction{concatenerCodeBinaireDansFichier(cbtemp,cb,fbcompresse)}
        		}
        			
        		
        \instruction{fermer(fbInitial)}
        \instruction{fermer(fbCompresse)}
        	
    }
\end{algorithme}
    	    			  			
    	    			  						   