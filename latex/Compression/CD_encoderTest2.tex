\begin{algorithme}
     \procedure{encoder}{\paramEntreeSortie{fbInitial : \fichierbinaire},\paramEntree{ tdc : \tabledecodage},\paramSortie{fbCompresse : \fichierbinaire}}
    {}
    {cbtemp, cb, restecbtemp : \codebinaire, oInitial, oCompresse : \octet, i,j, positionBit, tailleOctet : Entier}
    {
        \affecter{fbInitial}{fichierBinaire("nom du fichier")}
        \instruction{ouvrir(fbInitial,lecture)}
        \affecter{fbCompresse}{fichierBinaire("nom du fichier")}
        \instruction{ouvrir(fbCompresse, ecriture)}
        \instruction{ecrireoctet(naturelVersOctet(obtenirTailleFichier(fbInitiale)),fbcompresse)}%% le premier octet represente la taille du fichier initale à utiliser ou non 
        \affecter{tailleOctet}{1}
         {
        \affecter{restecbtemp}{creerCodeBinaire(0)}
        \pour{i}{1}{7}{}{\instruction{ajouterBit(restecbtemp,0)}}
        }
        \tantque{$non$ finfichier(fnInitiale)}{
       		\tantque{tailleOctet $<$ 8}{
       			
        		\instruction{lireoctet(fbInitiale,oInitial)}
        		\affecter{cb}{octetVersCodeBinaire(tdc,oInitial)}
        		\sialorssinon{obtenirlongeur(restecbtemp $>$ 1) $ET$ obtenirlongeur(restecbtemp $<$ 8)}
        		{
        				\affecter{cbtemp}{restecbtemp}
        		}
        		{		\instruction{creeCodeBinaire(obtenirIemeBit(cb,0)}
        		}
        		\affecter{positionBit}{obtenirLongeur(cbtemp)}
        		\pour{i}{positionBit $+$ 1}{obtenirLongeur(cbtemp)}{}
        		{
        					\instruction{ajouterBit(cbtemp,obtenirIemeBit(cb,i)}
        		}
        		\affecter{tailleOctet}{tailleOctet $+$ obtenirLongeur(cb)}
        	}
        	
        	\sialorssinon{(positionBit $<$ 7) $ET$ (positionBit $>$ 0)}
        	{
        		\affecter{i}{0}
        		\tantque{(positionBit $<$ 7) $ET$ (i $<$ obtenirLongeur(cb)}{
        			\instruction{ajouterBit(cbtemp,obtenirIemeBit(cb,i))}
        			\affecter{i}{i $+$ 1}
        		}
        			\affecter{oCompresse}{creeOctet(obtenirIemeBit(cbtemp,0),obtenirIemeBit(cbtemp,1),obtenirIemeBit(cbtemp,2),obtenirIemeBit(cbtemp,3), obtenirIemeBit(cbtemp,4),obtenirIemeBit(cbtemp,5),obtenirIemeBit(cbtemp,6),obtenirIemeBit(cbtemp,7))}
        	}
        	{\sialors{positionBit $=$ 7}{\affecter{oCompresse}{creeOctet(obtenirIemeBit(cbtemp,0),obtenirIemeBit(cbtemp,1),obtenirIemeBit(cbtemp,2),obtenirIemeBit(cbtemp,3), obtenirIemeBit(cbtemp,4),obtenirIemeBit(cbtemp,5),obtenirIemeBit(cbtemp,6),obtenirIemeBit(cbtemp,7))}}
        	}
        \instruction{ecrireOctet(fbcompresse,oCompresse)}
        \sialorssinon{obtenirlongeur(cb)$>$i}
        {
        	\affecter{restecbtemp}{creerCodeBinaire(obtenirIemeBit(cb,i))}
        	\pour{j}{i$+$1}{obtenirLongeur(cb)}{}{\instruction{ajouterbit(cbtemp,obtenirIemeBit(cb,j)}}
        }
        {
        \affecter{restecbtemp}{creerCodeBinaire(0)}
        \pour{i}{1}{7}{}{\instruction{ajouterBit(restecbtemp,0)}}
        }
 	}	
        			
        		
        \instruction{fermer(fbInitial)}
        \instruction{fermer(fbCompresse)}
        	
    }
\end{algorithme}
    	    			  			
    	    			  						   