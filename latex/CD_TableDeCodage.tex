\section{Conception Détaillée : TableDeCodage}

\newcommand{\octet}{\textbf{Octet}}
\newcommand{\codebinaire}{\textbf{CodeBinaire}}
\newcommand{\tabledecodage}{\textbf{TableDeCodage}}

\begin{algorithme}
    \type{TableDeCodage}{Dictionnaire<Octet,CodeBinaire>}
    \\
    \fonction{creerTableCodage}{o : \octet, cb : \codebinaire}{\tabledecodage}
    {}
    {tdc : \tabledecodage}
    {
        \affecter{tdc}{dictionnaire()}
        \instruction{ajouter(tdc,o,cb)}
        \retourner{tdc}
    }
    \\
    \procedure{ajouterCodage}{tdc : \tabledecodage, o : \octet, cb : \codebinaire}
    {non(octetPresent(t, octet))}
    {}
    {
        \instruction{ajouter(tdc,o,cb)}
    }
    \\
    \fonction{octetVersCodeBinaire}{tdc : \tabledecodage, o : \octet}{\codebinaire}
    {octetPresent(t, octet)}
    {}
    {
        \retourner{obtenirValeur(tdc,o)}
    }
    \\
    \fonction{octetPresent}{tdc : \tabledecodage, o : \octet}{\booleen}
    {}
    {}
    {
        \retourner{estPresent(tdc,o)}
    }
\end{algorithme}