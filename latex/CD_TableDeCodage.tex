\subsection{Conception Détaillée : TableDeCodage}

\newcommand{\octet}{\textbf{Octet}}
\newcommand{\codebinaire}{\textbf{CodeBinaire}}
\newcommand{\tabledecodage}{\textbf{TableDeCodage}}

\begin{algorithme}
    \type{TableDeCodage}{\tableauUneDimension{0..255}{de }{\codebinaire}}
    \\
    \fonction{creerTableCodage}{}{\tabledecodage}
    {}
    {tdc : TableDeCodage}
    {   \pour{octet}{ 0}{255}{}{
            \affecter{tdc[octet]}{creerCodeBinaire()}
        }
        \retourner{tdc}
    }
    \\
    \procedure{ajouterCodage}{tdc : \tabledecodage, o : \octet, cb : \codebinaire}
    {non(octetPresent(t, octet))}
    {}
    {
        \affecter{tdc[octetVersNaturel(o)]}{cb}
    }
    \\
    \fonction{octetVersCodeBinaire}{tdc : \tabledecodage, o : \octet}{\codebinaire}
    {octetPresent(t, octet)}
    {}
    {
        \retourner{tdc[octetVersNaturel(o)]}
    }
    \\
    \fonction{octetPresent}{tdc : \tabledecodage, o : \octet}{\booleen}
    {}
    {}
    {
        \retourner{non(estVide(tdc[octetVersNaturel(o)]))}
    }
\end{algorithme}
