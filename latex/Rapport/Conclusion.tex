Conclusion

Dans ce projet, nous avons développé un algorithme de compression par perte basé sur un arbre de Huffman. L'algorithme a été développé en 4 phases principales :

   \begin{enumerate}
\item La conception des TADs et des analyses descendantes de notre compresseur Huffman
\item La conception préliminaire de nos fonctions et procédures ;
\item La conception Détaillée.
\item Le code C et les tests unitaires;
\end{enumerate}

Nous avons évalué l'efficacité de l'algorithme de compression en le testant sur différents ensembles de données. Les résultats des tests montrent que l'algorithme est capable de réduire la taille des données de manière significative, sans perte de donnée.

Le travail de groupe a été essentiel à la réussite de ce projet. Nous avons pu partager les tâches et les responsabilités, bénéficier des compétences et des connaissances des autres membres du groupe, et résoudre les problèmes de manière collective.

les compétences et les connaissances acquises dans ce projet nous seront utile dans la poursuite de nos .

Quant aux perspectives d'amélioration l'algorithme de compression développé dans ce projet peut être amélioré de plusieurs manières, comme utiliser un autre algorithme pour construire l'arbre de Huffman ou encore s'interesser à des techniques de compression plus avancées, telles que la compression par blocs ou la compression par transformation et étudier les effets de la perte de qualité sur les données compressées.
