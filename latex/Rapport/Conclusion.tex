Conclusion

Dans ce projet, nous avons élaboré un algorithme de compression par perte basé sur un arbre de Huffman, en suivant une conception en quatre phase:

    La première phase a consisté en la conception des TADs (Types Abstraits de Données) et des analyses descendantes nécessaires à la mise en place de notre compresseur Huffman.

    Ensuite, nous avons réalisé la conception préliminaire de nos fonctions et procédures, jetant ainsi les bases de l'implémentation.

    La troisième phase a été consacrée à la conception détaillée, où nous avons approfondi chaque aspect de l'algorithme, clarifiant les spécifications et les interactions entre les différentes parties du code.

    Enfin, nous avons procédé à l'implémentation du code en langage C et à la réalisation des tests unitaires pour évaluer le fonctionnement des algorithmes.

Les résultats des tests, principalement réalisés sur des fichiers textes, démontrent l'efficacité de notre algorithme en termes de réduction de la taille des données, tout en préservant l'intégrité des informations d'origine.

L'utilisation d'une compression sans perte de données pour les fichiers texte est cruciale, car elle garantit la conservation du sens du texte d'origine. Nous aurions également pu explorer des méthodes de compression par perte de données et discuter de leur utilisation sur des fichiers audio ou images sur lesquels notre algorithme ne fonctionne pas étant des fichiers déjà compressé.

Le travail de groupe a été essentiel à la réussite de ce projet. Nous avons pu partager les tâches et les responsabilités, bénéficier des compétences et des connaissances des autres membres du groupe, et résoudre les problèmes de manière collective.

Concernant les perspectives d'amélioration, notre algorithme de compression pourrait être optimisé de différentes manières. Par exemple, l'utilisation d'un autre algorithme pour la construction de l'arbre de Huffman ou l'exploration de techniques de compression plus avancées telles que la compression par blocs ou la compression par transformation.


