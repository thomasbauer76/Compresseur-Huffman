\documentclass[10pt]{article}
\usepackage[utf8]{inputenc} % encodage du fichier
\usepackage[french]{babel} % document en francais
\usepackage[T1]{fontenc} % pour les caracteres accentues
\usepackage{geometry}
\geometry{a4paper, left=2.5cm, right=2.5cm, top=2.5cm, bottom=2.5cm}
\usepackage{pseudocode}
\usepackage{mdframed}

\begin{document}

\section{TAD ArbreDeHuffman}

%\begin{mdframed}
\begin{tad}
  \tadNom{CodeBinaire}
  \tadDependances{\textbf{Octet}, \naturel, \textbf{Bit} }
  \begin{tadOperations}{obtenirLongueur}
  
    \tadOperation{creeCodeBinaire}{bit}{CodeBinaire}
    \tadOperation{ajouterBit}{\tadParams{CodeBinaire, Bit}}{CodeBinaire}
    \tadOperation{retirerBit}{\tadParams{CodeBinaire, Bit}}{CodeBinaire}
    \tadOperationAvecPreconditions{obtenirBit}{\tadParams{CodeBinaire,\naturel}}{Bit}
    \tadOperation{obtenirLongueur}{CodeBinaire}{\naturel}
    \tadOperation{concatener}{\tadParams{CodeBinaire,CodeBinaire}}{CodeBinaire}
     
  \end{tadOperations}
  \parbox{\linewidth}{\raggedright
      \begin{tadAxiomes}
            \tadAxiome{ObtenirLongueur(concatener(CodeBinaire1,CodeBinaire2))=ObtenirLongueur(CodeBinaire1,CodeBinaire2)}
            \\
            \tadAxiome{ObtenirLongueur(ajouterBit(CodeBinaire,Bit))=ObtenirLongueur(CodeBinaire)+1}
      \end{tadAxiomes}
  }
  
  \begin{tadPreconditions}{ObtenirBit(CodeBinaire)}
    \tadPrecondition{obtenirBit(CodeBinaire,Naturel)}{Non(EstVide(CodeBinaire))}
  \end{tadPreconditions}
  
\end{tad}
%\end{mdframed}




\end{document}