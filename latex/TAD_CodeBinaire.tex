\subsection{TAD CodeBinaire}

%\begin{mdframed}
\begin{tad}
  \tadNom{CodeBinaire}
  \tadDependances{\textbf{Octet}, \naturel, \textbf{Bit} }
  \begin{tadOperations}{obtenirLongueur}
  
    \tadOperation{creerCodeBinaire}{Bit}{CodeBinaire}
    \tadOperation{ajouterBit}{\tadParams{CodeBinaire, Bit}}{CodeBinaire}
    \tadOperationAvecPreconditions{obtenirIemeBit}{\tadParams{CodeBinaire,\naturel}}{Bit}
    \tadOperation{obtenirLongueur}{CodeBinaire}{\naturel}
     
  \end{tadOperations}


  \begin{tadPreconditions}{ObtenirBit(CodeBinaire)}
    \tadPrecondition{obtenirIemeBit(cb : CodeBinaire, i : Naturel)}{i <= obtenirLongueur(cb)}
  \end{tadPreconditions}
    
  
  \begin{tadAxiomes}	
  	    \tadAxiome{obtenirLongueur(concatener(CodeBinaire1,CodeBinaire2))=
  		obtenirLongueur(CodeBinaire1,CodeBinaire2)}
  		\tadAxiome{obtenirLongueur(ajouterBit(CodeBinaire,Bit))=obtenirLongueur(CodeBinaire)+1}
   \end{tadAxiomes}
  
  	\begin{tadSemantiques}{ajouterBit}
  		\tadSemantique{ajouterBit}{Ajoute le bit en question à la fin du CodeBinaire}
  	\end{tadSemantiques}
\end{tad}
%\end{mdframed}
