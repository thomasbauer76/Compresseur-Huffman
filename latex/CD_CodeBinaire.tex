\subsection{Conception Détaillée : CodeBianire}

\begin{algorithme}
    \begin{enregistrement}{CodeBinaire}
        \champEnregistrement{tableBit}{\tableauUneDimension{0..7}{de}{\textbf{Bit}}}
        \champEnregistrement{nbElements}{naturel}
    \end{enregistrement}
    \\
    \fonction{creeCodeBinaire}
    {b: \textbf{Bit}}
    {\textbf{CodeBinaire}}
    {}
    {cb : \textbf{CodeBinaire}}
    {  		
    	\affecter{cb.tablebit[0]}{b}
    	\affecter{cb.nbElements}{1}
        \retourner{cb}
    }
    \\
    \fonction{obtenirIemeBit}
    {cb : \textbf{CodeBinaire}, i: \textbf{Naturel}}
    {\textbf{Bit}}
    {i<=obtenirLongueur(cb)}
    {}
    {
        \retourner{cb.tableBit[i]}
    }
 	\\
    \fonction{obtenirLongueur}{cb : CodeBinaire}
    {\textbf{Naturel}}
    {}
    {}
    {
       \retourner{cb.nbElements}
    }
    \\
     \procedure{ajouterBit}{E/S cb : CodeBinaire, E b : bit}
    {}
    {}
    {
       \affecter {cb.tableBit[obtenirLongueur(cd)+1}{b}
	    \affecter {cb.nbElements}{obtenirLongueur(cd)+1}
    }
\end{algorithme}
