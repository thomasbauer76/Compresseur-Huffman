\section{Analyse : les TAD}
            \subsection{TAD Octet}
                \subsection{TAD Octet}

%\begin{mdframed}
\begin{tad}
    \tadNom{Octet}
    \tadDependances{\textbf{Bit}, \textbf{0..7}}
    \begin{tadOperations}{octetVersNaturel}
        \tadOperation{creerOctet}{\tadParams{\textbf{Bit},\textbf{Bit},\textbf{Bit},\textbf{Bit},\textbf{Bit},\textbf{Bit},\textbf{Bit},\textbf{Bit}}}{\textbf{Octet}}
        \tadOperation{obtenirIemeBit}{\tadParams{\textbf{Octet},\textbf{0..7}}}{\textbf{Bit}}
        \tadOperation{octetVersNaturel}{\textbf{Octet}}{\textbf{0..255}}    
    \end{tadOperations}

    \begin{tadAxiomes}
        \tadAxiome{obtenirIemeBit(creerOctet(b_0,b_1,b_2,b_3,b_4,b_5,b_6,b_7),0)= b_0}\\
        \textit{Note : on obtient $b_1$ pour l'indice 1, ... et $b_7$ pour l'indice 7}
    \end{tadAxiomes}
    
    \begin{tadSemantiques}{octetVersNaturel}
    	\tadSemantique{octetVersNaturel}{permet la conversion d'un nombre en base 2 (Octet) vers un nombre en base 10 (Naturel)}
    \end{tadSemantiques}

\end{tad}
%\end{mdframed}

            \subsection{TAD Statistiques}
                \begin{tad}
  \tadNom{Statistiques}
  \tadDependances{\octet}
  \begin{tadOperations}{obtenirElementEtDefiler}
  
    \tadOperation{statistiques}{}{\statistiques}
    \tadOperation{incrementerOccurrence}{\tadParams{\statistiques, \octet}}{\statistiques}
    \tadOperation{obtenirOccurrence}{\tadParams{\statistiques, \octet}}{\naturel}
    \tadOperation{fixerOccurrence}{\tadParams{\statistiques, \octet, \naturel}}{\statistiques}
  
    %\tadOperation{longueur}{FileDePriorite}{\naturel}
    
  \end{tadOperations}

  \begin{tadAxiomes}
        \tadAxiome{obtenirOccurrence(statistique(), o)=0}
        \tadAxiome{obtenirOccurrence(incrementerOccurrence(s,o), o) = obtenirOccurrence(s,o)+1}
 		\tadAxiome{obtenirOccurrence(fixerOccurrence(s, o, n), o) = n}
  \end{tadAxiomes}
  
  
  
\end{tad}
            \subsection{TAD FileDePriorite}
                \documentclass[10pt]{article}
\usepackage[utf8]{inputenc} % encodage du fichier
\usepackage[french]{babel} % document en francais
\usepackage[T1]{fontenc} % pour les caracteres accentues
\usepackage{geometry}
\geometry{a4paper, left=2.5cm, right=2.5cm, top=2.5cm, bottom=2.5cm}
\usepackage{pseudocode}
\usepackage{mdframed}

\begin{document}

%\begin{mdframed}
\begin{tad}
  \tadNom{FileDePriorite}
  \tadParametres{Element ($\forall e_1 \in Element, e_2 \in Element, e_1 \neq e_2 \Rightarrow e_1 < e_2 \textit{ ou } e_1 > e_2$)}
  \tadDependances{\booleen}
  \begin{tadOperations}{obtenirElementEtDefiler}
  
    \tadOperation{fileDePriorite}{}{FileDePriorite}
    \tadOperation{estVide}{FileDePriorite}{\booleen}
    \tadOperation{enfiler}{\tadParams{FileDePriorite, Element}}{FileDePriorite}
    \tadOperationAvecPreconditions{obtenirElementEtDefiler}{FileDePriorite}{\tadParams{FileDePriorite, Element}}
    %\tadOperation{longueur}{FileDePriorite}{\naturel}
    
  \end{tadOperations}
  \parbox{\linewidth}{\raggedright
      \begin{tadAxiomes}
          \tadAxiome{estVide(fileDePriorite())}
          \tadAxiome{non(estVide(enfiler(f, e)))}
          \tadAxiome{obtenirElementEtDefiler(enfiler(fileDePriorite(),e)) = fileDePriorite(),e}
          \tadAxiome{e \leq obtenirElementEtDefiler(f)[2] \Rightarrow obtenirElementEtDefiler(enfiler(f,e)) = f,e}
          \tadAxiome{e > obtenirElementEtDefiler(f)[2] \Rightarrow obtenirElementEtDefiler(enfiler(f,e)) = enfiler(obtenirElementEtDefiler(f)[1],e),obtenirElementEtDefiler(f)[2]}
      \end{tadAxiomes}
  }
  
  \begin{tadPreconditions}{obtenirElementEtDefiler(f)}
    \tadPrecondition{obtenirElementEtDefiler(f)}{non(estVide(f))}
  \end{tadPreconditions}
  
\end{tad}
%\end{mdframed}




\end{document}

            \subsection{TAD ArbreDeHuffman}
                \documentclass[10pt]{article}
\usepackage[utf8]{inputenc} % encodage du fichier
\usepackage[french]{babel} % document en francais
\usepackage[T1]{fontenc} % pour les caracteres accentues
\usepackage{geometry}
\geometry{a4paper, left=2.5cm, right=2.5cm, top=2.5cm, bottom=2.5cm}
\usepackage{pseudocode}
\usepackage{mdframed}

\begin{document}

%\begin{mdframed}
\begin{tad}
  \tadNom{ArbreDeHuffman}
  \tadParametres{Element, Frequence}
  \tadDependances{\booleen}
  \begin{tadOperations}{obtenirFilsGauche}
  
    \tadOperation{arbreDeHuffman}{\tadParams{Element, Frequence}}{ArbreDeHuffman}
    \tadOperation{ajouterRacine}{\tadParams{ArbreDeHuffman, ArbreDeHuffman}}{ArbreDeHuffman}
    \tadOperation{estUneFeuille}{ArbreDeHuffman}{\booleen}
    \tadOperationAvecPreconditions{obtenirElement}{ArbreDeHuffman}{Element}
    \tadOperation{obtenirFrequence}{ArbreDeHuffman}{Frequence}
    \tadOperationAvecPreconditions{obtenirFilsGauche}{ArbreDeHuffman}{ArbreDeHuffman}
    \tadOperationAvecPreconditions{obtenirFilsDroit}{ArbreDeHuffman}{ArbreDeHuffman}
    
  \end{tadOperations}
  \parbox{\linewidth}{\raggedright
      \begin{tadAxiomes}
            \tadAxiome{estUneFeuille(arbreDeHuffman(e,f))}
            \tadAxiome{non(estUneFeuille(ajouterRacine(a_g,a_d)))}
            \tadAxiome{obtenirElement(arbreDeHuffman(e,f)) = e}
            \tadAxiome{obtenirFrequence(arbreDeHuffman(e,f)) = f}
            \tadAxiome{obtenirFrequence(ajouterRacine(a_g,a_d)) = obtenirFrequence(a_g) + obtenirFrequence(a_d)}
            \tadAxiome{obtenirFilsGauche(ajouterRacine(a_g,a_d)) = a_g}
            \tadAxiome{obtenirFilsDroit(ajouterRacine(a_g,a_d)) = a_d}
      \end{tadAxiomes}
  }
  
  \begin{tadPreconditions}{obtenirFilsGauche(a)}
    \tadPrecondition{obtenirElement(a)}{estUneFeuille(a)}
    \tadPrecondition{obtenirFilsGauche(a)}{non(estUneFeuille(a))}
    \tadPrecondition{obtenirFilsDroit(a)}{non(estUneFeuille(a))}
  \end{tadPreconditions}
  
\end{tad}
%\end{mdframed}

\end{document}

            \subsection{TAD CodeBinaire}
                \begin{tad}
  \tadNom{CodeBinaire}
  \tadDependances{\textbf{Octet}, \naturel, \textbf{Bit}}
  \begin{tadOperations}{obtenirLongueur}
  
    \tadOperation{creerCodeBinaire}{\bit}{\codebinaire}
    \tadOperation{ajouterBit}{\tadParams{\codebinaire, \bit}}{\codebinaire}
    \tadOperationAvecPreconditions{obtenirIemeBit}{\tadParams{\codebinaire, \naturel}}{\bit}
    \tadOperation{obtenirLongueur}{\codebinaire}{\naturel}
     
  \end{tadOperations}

  \begin{tadPreconditions}{obtenirIemeBit(cb, i)}
    \tadPrecondition{obtenirIemeBit(cb, i)}{i < obtenirLongueur(cb)}
  \end{tadPreconditions}

  \begin{tadAxiomes}	

    \tadAxiome{obtenirLongueur(creerCodeBinaire(b)) = 1}
    \tadAxiome{obtenirLongueur(ajouterBit(cb, b)) = obtenirLongueur(cb) + 1}

  \end{tadAxiomes}
  
  \begin{tadSemantiques}{obtenirIemeBit}

    \tadSemantique{ajouterBit}{Ajoute le bit en question à la fin du CodeBinaire}
    \tadSemantique{obtenirIemeBit}{Retourne le bit à la position i du CodeBinaire (0 étant la position du premier bit)}
    
  \end{tadSemantiques}
\end{tad}
            \subsection{TAD TableDeCodage}
                \begin{tad}
    \tadNom{TableDeCodage}
    \tadDependances{\textbf{Octet}, \booleen, \textbf{CodeBinaire}}
    \begin{tadOperations}{octetVersCodeBinaire}
        \tadOperation{creerTableCodage}{}{\textbf{TableDeCodage}}
        \tadOperationAvecPreconditions{ajouterCodage}{\tadParams{\textbf{TableDeCodage},\textbf{Octet},\textbf{CodeBinaire}}}{\textbf{TableDeCodage}}
        \tadOperation{octetPresent}{\tadParams{\textbf{TableDeCodage},\textbf{Octet}}}{\textbf{\booleen}}
        \tadOperationAvecPreconditions{octetVersCodeBinaire}{\tadParams{\textbf{TableDeCodage},\textbf{Octet}}}{\textbf{CodeBinaire}}
    \end{tadOperations}

    \begin{tadPreconditions}{octetVersCodeBinaire(t, codeBinaire)}
        \tadPrecondition{ajouterCodage(t,octet,codeBinaire)}{non(octetPresent(t, octet))}
        \tadPrecondition{octetVersCodeBinaire(t, octet)}{octetPresent(t, octet)}
    \end{tadPreconditions}

    \begin{tadAxiomes}
    	\tadAxiome{octetPresent(ajouterCodage(t, octet, codeBinaire), octet)}
    	\tadAxiome{non(octetPresent(creerTableCodage(), octet))}
        \tadAxiome{octetVersCodeBinaire(ajouterCodage(t, octet, codeBinaire), octet)=codeBinaire}
    \end{tadAxiomes}
    

\end{tad}


        \newpage
        \section{Conception Préliminaire}
            \subsection{Signatures et fonctions de Octet}
                \subsection{Conception préliminaire : Octet}

\begin{algorithme}
    \signatureFonction{creerOctet}
    {b0, b1, b2, b3, b4, b5, b6, b7 : \textbf{Bit}}
    {\textbf{Octet}}
    {}
    \signatureFonction{obtenirIemeBit}
    {o : \textbf{Octet}, b : \textbf{0..7}}
    {\textbf{Bit}}
    {}
    \signatureFonction{octetVersNaturel}
    {o : \textbf{Octet}}
    {\textbf{0..255}}
    {}
    \signatureFonction{naturelVersOctet}
    {n : \textbf{0..255}}
    {\textbf{Octet}}
    {}
\end{algorithme}

            \subsection{Signatures et fonctions de Statistiques}
                \subsection{Conception préliminaire : Statistiques}

\begin{algorithme}
    \signatureFonction{statistiques}
    {}{\textbf{Statistiques}}
    {}
    \signatureProcedure{incrementerOccurence}
    {\paramEntreeSortie{s : \textbf{Statistiques}}, o : \paramEntree{\textbf{Octet}}}
    {}
    \signatureFonction{obtenirOccurence}
    {o : \textbf{Octet}, s : \textbf{Statistiques}}
    {\naturel}
    {}
\end{algorithme}
            \subsection{Signatures et fonctions de FileDePriorite}
                \begin{algorithme}
    \signatureFonction{fileDePriorite}{}{FileDePriorite}{}
    \signatureFonction{estVide}{fdp : FileDePriorite}{Booléen}{}
    \signatureProcedure{enfiler}{E/S fdp : FileDePriorite, E e : Element}{}
    \signatureProcedure{obtenirElementEtDefiler}{E/S fdp : FileDePriorite, S Element}{non(estVide(fdp))}
\end{algorithme}
            \subsection{Signatures et fonctions de ArbreDeHuffman}
                \subsection{Conception préliminaire : ArbreDeHuffman}

\begin{algorithme}

    \signatureFonction{arbreDeHuffman}
    {o : \textbf{Octet}, n : \textbf{Naturel}}
    {\textbf{ArbreDeHuffman}}{}
    \signatureFonction{fusionner}{ag : ArbreDeHuffman, ad : ArbreDeHuffman} {\textbf{ArbreDeHuffman}}{}
    
    \signatureFonction{estUneFeuille}{a: ArbreDeHuffman}
    {\textbf{Booleen}}{}
    \signatureFonction{obtenirOctect}
    {a: ArbreDeHuffman}
    {\textbf{Octet}}{estUneFeuille(a)}
    \signatureFonction{obtenirFrequence}
    {a: ArbreDeHuffman}
    {\textbf{Naturel}}{}
     \signatureFonction{obtenirFilsGauche}
    {a: ArbreDeHuffman}
    {\textbf{ArbreDeHuffman}}{non(estUneFeuille(a))}
      \signatureFonction{obtenirFilsDroit}
    {a: ArbreDeHuffman}
    {\textbf{ArbreDeHuffman}}{non(estUneFeuille(a))}
    
\end{algorithme}

            \subsection{Signatures et fonctions de CodeBinaire}
                \begin{algorithme}
    \signatureFonction{creerCodeBinaire}{b : \bit}{\codebinaire}{}
    \signatureProcedure{ajouterBit}{\paramEntreeSortie{cb : \codebinaire}, \paramEntree{b : \bit}}{}
    \signatureFonction{obtenirIemeBit}{cb : \codebinaire, i : \naturel}{\bit}{i < obtenirLongueur(cb)}
    \signatureFonction{obtenirLongueur}{cb : \codebinaire}{\naturel}{}
\end{algorithme}

            \subsection{Signatures et fonctions de TableDeCodage}
                \begin{algorithme}
    \signatureFonction{creerTableCodage}{}{TableDeCodage}{}
    \signatureProcedure{ajouterCodage}{E/S tdc : TableDeCodage, E o : octet, E cb:Codebinaire}{}
     \signatureFonction{OctetPresent}{tdc : TableDeCodage,o :octet}{Booleen}{}
    \signatureFonction{OctetVersCodeBinaire}{tdc : TableDeCodage,o :octet}{CodeBinaire}{octetPresent(t,octet)}
   
\end{algorithme}

        \newpage
        \section{Conception Détaillée}
            \textbf{// Dans le but de simplifier la conception détaillée ainsi que le développement, nous considérons bitA0 = 0 et bitA1 = 1. Nous pouvons ainsi utiliser le type Bit comme un naturel. Cela permet d'éviter des instructions conditionnelles répétitives et coûteuses bien que triviales.}
            \subsection{Algorithmes de Octet}
                \begin{algorithme}
    \type{Octet}{\textbf{0..255}}
    \\
    \fonction{creerOctet}
    {b7, b6, b5, b4, b3, b2, b1, b0 : \textbf{Bit}}
    {\textbf{Octet}}
    {}
    {o : \textbf{Octet}}
    {  		
        \affecter{o}{b0 + b1*2 + b2*2\^{}2 + b3*2\^{}3 + b4*2\^{}4 + b5*2\^{}5 + b6*2\^{}6 + b7*2\^{}7}
        \retourner{o}
    }
    \\
    \fonction{obtenirIemeBit}
    {o : \textbf{Octet}, b : \textbf{0..7}}
    {\textbf{Bit}}
    {}
    {}
    {
        \retourner{(o \textbf{div} 2\^{}b) \textbf{mod} 2}
    }
 	\\
    \fonction{octetVersNaturel}
    {o : \textbf{Octet}}
    {\textbf{0..255}}
    {}
    {}
	{   
    {
       \retourner{o}
    }
    }
    \\
    \fonction{naturelVersOctet}
    {n : \textbf{0..255}}
    {\textbf{Octet}}
    {}
    {}
	{   
    {
       \retourner{n}
    }
    }
\end{algorithme}

            \subsection{Algorithmes de Statistiques}
                \begin{algorithme}
    \type{Statistiques}{\tableauUneDimension{0..255}{de }{\naturel}}
    \\
    \fonction{statistiques}{}{\statistiques}
    {}
    {s : \statistiques}
    {
        \pour{octet}{ 0}{255}{}{
            \affecter{s[octet]}{0}
        }
        \retourner{s}
    }
    \\
    \procedure{incrementerOccurrence}{\paramEntreeSortie{s : \statistiques}, \paramEntree{o : \octet}}
    {}
    {}
    {
    	\affecter{s[octetVersNaturel(o)]}{s[octetVersNaturel(o)] + 1}
    }
    \\
    \fonction{obtenirOccurrence}{s : \statistiques, o : \octet}{\textbf{Naturel}}
    {}
    {}
    {
    	\retourner{s[octetVersNaturel(o)]}
    }
    \procedure{fixerOccurrence}{\paramEntreeSortie{s : \statistiques}, \paramEntree{o : \octet, n : \naturel}}
    {}
    {}
    {
    	\affecter{s[octetVersNaturel(o)]}{n}
    }
\end{algorithme}
            \subsection{Algorithmes de FileDePriorite}
                \subsection{File de Priorité}
\begin{algorithme}
    \procedure{enfiler}{\paramEntreeSortie{file : \textbf{FileDePriorite}}, \paramEntree{e : \textbf{Element}}}{}
    {temp : \textbf{FileDePriorite}}
    {
         \sialorssinon{estVide(file) ou e $\leq$ file\motclefDereferencer .element}
         {
            \affecter{temp}{file}
            \instruction{\allouer{file}}
            \affecter{file\motclefDereferencer .element}{e}
            \affecter{file\motclefDereferencer .fileSuivante}{temp}
         }
         {
            \instruction{\textbf{enfiler}(file\motclefDereferencer .fileSuivante, e)}{}
         }
    }
\end{algorithme}

\bigbreak
\begin{algorithme}
    \procedure{obtenirElementEtDefiler}{\paramEntreeSortie{file : \textbf{FileDePriorite}, \paramSortie{e : \textbf{Element}}}}{non(estVide(file))}
    {temp : \textbf{FileDePriorite}}
    {
        \affecter{e}{file\motclefDereferencer .element}
        \affecter{temp}{file}
        \affecter{file}{temp\motclefDereferencer .listeSuivante}
        \instruction{\desallouer{temp}}
    }
\end{algorithme}

\bigbreak
\begin{algorithme}
    \fonction{fileDePriorite}{}{\textbf{FileDePriorite}}{}
    {}
    {
        \retourner{\textbf{NIL}}
    }
\end{algorithme}

\bigbreak
\begin{algorithme}
    \fonction{estVide}{file : \textbf{FileDePriorite}}{\booleen}{}
    {}
    {
        \retourner{file = \textbf{NIL}}
    }
\end{algorithme}
            \subsection{Algorithmes de ArbreDeHuffman}
                \begin{algorithme}
	\type{ArbreDeHuffman}{$\widehat{}$ Noeud}
	\begin{enregistrement}{Noeud}
		\champEnregistrement{octet}{\octet}
		\champEnregistrement{frequence}{\naturel}
		\champEnregistrement{estUneFeuille}{\booleen}
		\champEnregistrement{filsGauche}{\arbredehuffman}
		\champEnregistrement{filsDroit}{\arbredehuffman}
	\end{enregistrement}
	\\
    \fonction{arbreDeHuffman}{o : \octet, f : \naturel}{\arbredehuffman}{}
    {a : ArbreDeHuffman}
    {
        \instruction{\allouer{a}}
        \affecter{a\motclefDereferencer .octet}{o}
        \affecter{a\motclefDereferencer .frequence}{f}
        \affecter{a\motclefDereferencer .estUneFeuille}{\textbf{Vrai}}
        \affecter{a\motclefDereferencer .filsGauche}{\textbf{NIL}}
        \affecter{a\motclefDereferencer .filsDroit}{\textbf{NIL}}
        \retourner{a}
    }
	\\
    \fonction{fusionner}{$a_g$, $a_d$ : \arbredehuffman}{\arbredehuffman}{}
    {racine : \arbredehuffman}
    {
        \instruction{\allouer{racine}}
        \affecter{racine\motclefDereferencer .filsGauche}{$a_g$}
        \affecter{racine\motclefDereferencer .filsDroit}{$a_d$}
        \affecter{racine\motclefDereferencer .estUneFeuille}{\textbf{Faux}}
        \affecter{racine\motclefDereferencer .frequence}{obtenirFrequence($a_g$) + obtenirFrequence($a_d$)}
        \retourner{racine}
    }
	\\
    \fonction{estUneFeuille}{a : \arbredehuffman}{\booleen}{}
    {}
    {
        \retourner{a\motclefDereferencer .estUneFeuille}
    }
	\\
    \fonction{obtenirOctet}{a : \arbredehuffman}{\octet}{estUneFeuille(a)}
    {}
    {
        \retourner{a\motclefDereferencer .octet}
    }
	\\
    \fonction{obtenirFrequence}{a : \arbredehuffman}{\naturel}{}
    {}
    {
        \retourner{a\motclefDereferencer .frequence}
    }
	\\
    \fonction{obtenirFilsGauche}{a : \arbredehuffman}{\arbredehuffman}{non(estUneFeuille(a))}
    {}
    {
        \retourner{a\motclefDereferencer .filsGauche}
    }
	\\
    \fonction{obtenirFilsDroit}{a : \arbredehuffman}{\arbredehuffman}{non(estUneFeuille(a))}
    {}
    {
        \retourner{a\motclefDereferencer .filsDroit}
    }
    \\
    \procedure{liberer}{\paramEntree{arbre : \arbredehuffman}}{}{}
    {
        \sialors{non(estUneFeuille(arbre))}
        {
            \instruction{liberer(obtenirFilsGauche(arbre))}
            \instruction{liberer(obtenirFilsDroit(arbre))}
        }
        \instruction{\textbf{desallouer}(arbre)}
    }
\end{algorithme}

            \subsection{Algorithmes de CodeBinaire}
                \begin{algorithme}
    \begin{enregistrement}{CodeBinaire}
        \champEnregistrement{codeBinaire}{\textbf{Naturel}}
        \champEnregistrement{nbBits}{\textbf{Naturel}}
    \end{enregistrement}
    \\
    \fonction{creeCodeBinaire}
    {b: \textbf{Naturel}}
    {\textbf{CodeBinaire}}
    {}
    {cb : \textbf{CodeBinaire}}
    {  		
    	\affecter{cb.codebinaire}{b}
    	\affecter{cb.nbBits}{1}
        \retourner{cb}
    }
    \\
    \fonction{obtenirIemeBit}
    {cb : \textbf{CodeBinaire}, i: \textbf{0..7}}
    {\textbf{Bit}}
    {}
    {}
    {
        \retourner{cb.codeBinaire/$2^{i}$}
    }
 	\\
    \fonction{obtenirLongueur}{cb : CodeBinaire}
    {\textbf{Naturel}}
    {}
    {}
    {
       \retourner{cb.bits}
    }
    \\
     \procedure{ajouterBit}{E/S cb : CodeBinaire, E b : bit}
    {}
    {}
    {
       \affecter {cb.codebinaire}{cb.codebinaire + (b*$2^{cb.nbBits}$)}
	    \affecter {cb.nbElements}{obtenirLongueur(cd)+1}
    }
\end{algorithme}

            \subsection{Algorithmes de TableDeCodage}
                \section{Conception Détaillée : TableDeCodage}

\newcommand{\octet}{\textbf{Octet}}
\newcommand{\codebinaire}{\textbf{CodeBinaire}}
\newcommand{\tabledecodage}{\textbf{TableDeCodage}}

\begin{algorithme}
    \type{TableDeCodage}{Dictionnaire<Octet,CodeBinaire>}
    \\
    \fonction{creerTableCodage}{o : \octet, cb : \codebinaire}{\tabledecodage}
    {}
    {tdc : \tabledecodage}
    {
        \affecter{tdc}{dictionnaire()}
        \instruction{ajouter(tdc,o,cb)}
        \retourner{tdc}
    }
    \\
    \procedure{ajouterCodage}{tdc : \tabledecodage, o : \octet, cb : \codebinaire}
    {non(octetPresent(t, octet))}
    {}
    {
        \instruction{ajouter(tdc,o,cb)}
    }
    \\
    \fonction{octetVersCodeBinaire}{tdc : \tabledecodage, o : \octet}{\codebinaire}
    {octetPresent(t, octet)}
    {}
    {
        \retourner{obtenirValeur(tdc,o)}
    }
    \\
    \fonction{octetPresent}{tdc : \tabledecodage, o : \octet}{\booleen}
    {}
    {}
    {
        \retourner{estPresent(tdc,o)}
    }
\end{algorithme}