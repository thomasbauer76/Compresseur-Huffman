\section{Analyse : les TAD}
            \subsection{TAD Octet}
                \documentclass[10pt]{article}
\usepackage[utf8]{inputenc} % encodage du fichier
\usepackage[french]{babel} % document en francais
\usepackage[T1]{fontenc} % pour les caracteres accentues
\usepackage{geometry}
\geometry{a4paper, left=2.5cm, right=2.5cm, top=2.5cm, bottom=2.5cm}
\usepackage{pseudocode}
\usepackage{mdframed}

\begin{document}

\section{TAD Table de Codage}

%\begin{mdframed}
\begin{tad}
    \tadNom{Octet}
    \tadDependances{\textbf{Bit}, \textbf{[1..8]}}
    \begin{tadOperations}{octetVersNaturel}
        \tadOperation{creerOctet}{\tadParams{\textbf{Bit},\textbf{Bit},\textbf{Bit},\textbf{Bit},\textbf{Bit},\textbf{Bit},\textbf{Bit},\textbf{Bit}}}{\textbf{Octet}}
        \tadOperation{obtenirIemeBit}{\tadParams{\textbf{Octet},\textbf{[1..8]}}}{\textbf{Bit}}
        \tadOperation{octetVersNaturel}{\textbf{Octet}}{\textbf{[1..256]}}
        \tadOperation{naturelVersOctet}{\textbf{[1..256]}}{\textbf{Octet}}
       
       
    \end{tadOperations}

    \begin{tadPreconditions}{octetVersCodeBinaire(t, codeBinaire)}
        \tadPrecondition{Aucune}
    \end{tadPreconditions}

    \begin{tadAxiomes}{}
        A mieux faire 	après avoir revu TOUTES les opérations du TAD
    \end{tadAxiomes}

\end{tad}
%\end{mdframed}




\end{document}


            \subsection{TAD Statistiques}
                \begin{tad}
  \tadNom{Statistiques}
  \tadDependances{CodeBinaire, Octet}
  \begin{tadOperations}{obtenirElementEtDefiler}
  
    \tadOperation{statistique}{}{Statistiques}
    \tadOperation{incrementerOccurence}{\tadParams{Statistiques, Octet}}{Statistiques}
    \tadOperation{obtenirOccurence}{\tadParams{Statistiques, Octet}}{Naturel}
  
    %\tadOperation{longueur}{FileDePriorite}{\naturel}
    
  \end{tadOperations}
  \parbox{\linewidth}{\raggedright
      \begin{tadAxiomes}
          \tadAxiome{obtenirOccurence(statistique(), octet)=0}
          \tadAxiome{obtenirOccurence(incrementerOccurence(stat,octet), octet) = obtenirOccurence(stat,octet)+1}
         
      \end{tadAxiomes}
  }
  
  
\end{tad}
            \subsection{TAD FileDePriorite}
                \documentclass[10pt]{article}
\usepackage[utf8]{inputenc} % encodage du fichier
\usepackage[french]{babel} % document en francais
\usepackage[T1]{fontenc} % pour les caracteres accentues
\usepackage{geometry}
\geometry{a4paper, left=2.5cm, right=2.5cm, top=2.5cm, bottom=2.5cm}
\usepackage{pseudocode}
\usepackage{mdframed}

\begin{document}

%\begin{mdframed}
\begin{tad}
  \tadNom{FileDePriorite}
  \tadParametres{Element ($\forall e_1 \in Element, e_2 \in Element, e_1 \neq e_2 \Rightarrow e_1 < e_2 \textit{ ou } e_1 > e_2$)}
  \tadDependances{\booleen}
  \begin{tadOperations}{obtenirElement}
  
    \tadOperation{fileDePriorite}{}{FileDePriorite}
    \tadOperation{estVide}{FileDePriorite}{\booleen}
    \tadOperation{enfiler}{\tadParams{FileDePriorite, Element}}{FileDePriorite}
    \tadOperationAvecPreconditions{obtenirElementEtDefiler}{FileDePriorite}{\tadParams{FileDePriorite, Element}}
    %\tadOperation{longueur}{FileDePriorite}{\naturel}
    
  \end{tadOperations}
  \parbox{\linewidth}{\raggedright
      \begin{tadAxiomes}
          \tadAxiome{estVide(fileDePriorite())}
          \tadAxiome{non(estVide(enfiler(f, e)))}
          \tadAxiome{obtenirElementEtDefiler(enfiler(fileDePriorite(),e)) = fileDePriorite(), e}
          \tadAxiome{e_2 \leq e_1 \Rightarrow obtenirElementEtDefiler(enfiler(enfiler(fileDePriorite(),e_1),e_2)) = enfiler(fileDePriorite(),e1),e2}
          \tadAxiome{e_2 > e_1 \Rightarrow obtenirElementEtDefiler(enfiler(enfiler(fileDePriorite(),e_1),e_2)) = enfiler(fileDePriorite(),e_2),e_1}
      \end{tadAxiomes}
  }
  
  \begin{tadPreconditions}{obtenirElement(f)}
    \tadPrecondition{obtenirElementEtDefiler(f)}{non(estVide(f))}
  \end{tadPreconditions}
  
\end{tad}
%\end{mdframed}





\section{Signatures des fonctions et procédures}

\begin{algorithme}
    \signatureFonction{fileDePriorite}{}{\textbf{FileDePriorite}}{}
    \signatureFonction{estVide}{file : \textbf{FileDePriorite}}{\booleen}{}
    \signatureProcedure{enfiler}{\paramEntreeSortie{file : \textbf{FileDePriorite}}, \paramEntree{e : \textbf{Element}}}{}
    \signatureProcedure{obtenirElementEtDefiler}{\paramEntreeSortie{file : \textbf{FileDePriorite}}, \paramSortie{e : \textbf{Element}}}{non(estVide(file))}
\end{algorithme}



\subsection{Type FileDePriorite}

\begin{algorithme}
    \type{FileDePriorite}{\motclefDereferencer Noeud}
    \bigbreak
    \begin{enregistrement}{Noeud}
        \champEnregistrement{element}{Element}
        \champEnregistrement{fileSuivante}{FileDePriorite}
    \end{enregistrement}
\end{algorithme}



\subsection{Algorithmes des fonctions et procédures}

\begin{algorithme}
    \fonction{fileDePriorite}{}{\textbf{FileDePriorite}}{}
    {}
    {
        \retourner{\textbf{NIL}}
    }
\end{algorithme}

\bigbreak
\begin{algorithme}
    \fonction{estVide}{file : \textbf{FileDePriorite}}{\booleen}{}
    {}
    {
        \retourner{file = \textbf{NIL}}
    }
\end{algorithme}

\bigbreak
\begin{algorithme}
    \procedure{enfiler}{\paramEntreeSortie{file : \textbf{FileDePriorite}}, \paramEntree{e : \textbf{Element}}}{}
    {temp : \textbf{FileDePriorite}}
    {
         \sialorssinon{estVide(file) ou e $\leq$ file\motclefDereferencer .element}
         {
            \affecter{temp}{file}
            \instruction{\allouer{file}}
            \affecter{file\motclefDereferencer .element}{e}
            \affecter{file\motclefDereferencer .fileSuivante}{temp}
         }
         {
            \instruction{\textbf{enfiler}(file\motclefDereferencer .fileSuivante, e)}{}
         }
    }
\end{algorithme}

\bigbreak
\begin{algorithme}
    \procedure{obtenirElementEtDefiler}{\paramEntreeSortie{file : \textbf{FileDePriorite}}, \paramSortie{e : \textbf{Element}}}{non(estVide(file))}
    {temp : \textbf{FileDePriorite}}
    {
        \affecter{e}{file\motclefDereferencer .element}
        \affecter{temp}{file}
        \affecter{file}{temp\motclefDereferencer .listeSuivante}
        \instruction{\desallouer{temp}}
    }
\end{algorithme}




\end{document}

            \subsection{TAD ArbreDeHuffman}
                \documentclass[10pt]{article}
\usepackage[utf8]{inputenc} % encodage du fichier
\usepackage[french]{babel} % document en francais
\usepackage[T1]{fontenc} % pour les caracteres accentues
\usepackage{geometry}
\geometry{a4paper, left=2.5cm, right=2.5cm, top=2.5cm, bottom=2.5cm}
\usepackage{pseudocode}
\usepackage{mdframed}

\begin{document}

\section{TAD ArbreDeHuffman}

%\begin{mdframed}
\begin{tad}
  \tadNom{ArbreDeHuffman}
  \tadDependances{\textbf{Statistiques}, \textbf{0..255}, \textbf{Octet}, \naturel, \booleen}
  \begin{tadOperations}{obtenirFilsGauche}
  
    \tadOperation{arbreDeHuffman}{\tadParams{\textbf{Statistiques},\textbf{0..255}}}{ArbreDeHuffman}
    \tadOperation{fusionner}{\tadParams{ArbreDeHuffman, ArbreDeHuffman}}{ArbreDeHuffman}
    \tadOperation{estUneFeuille}{ArbreDeHuffman}{\booleen}
    \tadOperationAvecPreconditions{obtenirOctet}{ArbreDeHuffman}{\textbf{Octet}}
    \tadOperation{obtenirFrequence}{ArbreDeHuffman}{\naturel}
    \tadOperationAvecPreconditions{obtenirFilsGauche}{ArbreDeHuffman}{ArbreDeHuffman}
    \tadOperationAvecPreconditions{obtenirFilsDroit}{ArbreDeHuffman}{ArbreDeHuffman}
    
  \end{tadOperations}
  \parbox{\linewidth}{\raggedright
      \begin{tadAxiomes}
            \tadAxiome{estUneFeuille(arbreDeHuffman(s,i))}
            \tadAxiome{non(estUneFeuille(ajouterRacine(a_g,a_d)))}
            \tadAxiome{obtenirOctet(arbreDeHuffman(s,i)) = Statistiques.obtenirOctet(s,i)}
            \tadAxiome{obtenirFrequence(arbreDeHuffman(s,i)) = Statistiques.obtenirFrequence(s,i)}
            \tadAxiome{obtenirFrequence(ajouterRacine(a_g,a_d)) = obtenirFrequence(a_g) + obtenirFrequence(a_d)}
            \tadAxiome{obtenirFilsGauche(ajouterRacine(a_g,a_d)) = a_g}
            \tadAxiome{obtenirFilsDroit(ajouterRacine(a_g,a_d)) = a_d}
      \end{tadAxiomes}
  }
  
  \begin{tadPreconditions}{obtenirFilsGauche(a)}
    \tadPrecondition{obtenirOctet(a)}{estUneFeuille(a)}
    \tadPrecondition{obtenirFilsGauche(a)}{non(estUneFeuille(a))}
    \tadPrecondition{obtenirFilsDroit(a)}{non(estUneFeuille(a))}
  \end{tadPreconditions}
  
\end{tad}
%\end{mdframed}

\section{Signatures des fonctions et procédures}

\begin{algorithme}
    \signatureFonction{arbreDeHuffman}{stat : \textbf{Statistiques}, octet : \textbf{0..255}}{\textbf{ArbreDeHuffman}}{}
    \signatureFonction{fusionner}{$a_g$, $a_d$ : \textbf{ArbreDeHuffman}}{\textbf{ArbreDeHuffman}}{}
    \signatureFonction{estUneFeuille}{arbre : \textbf{ArbreDeHuffman}}{\booleen}{}
    \signatureFonction{obtenirOctet}{arbre : \textbf{ArbreDeHuffman}}{\textbf{Octet}}{estUneFeuille(arbre)}    
    \signatureFonction{obtenirFrequence}{arbre : \textbf{ArbreDeHuffman}}{\naturel}{}
    \signatureFonction{obtenirFilsGauche}{arbre : \textbf{ArbreDeHuffman}}{\textbf{ArbreDeHuffman}}{non(estUneFeuille(arbre))}
    \signatureFonction{obtenirFilsDroit}{arbre : \textbf{ArbreDeHuffman}}{\textbf{ArbreDeHuffman}}{non(estUneFeuille(arbre))}
\end{algorithme}

\section{Conception détaillée}

\subsection{Type ArbreDeHuffman}

\begin{algorithme}
    \type{ArbreDeHuffman}{\motclefDereferencer Racine}
    \bigbreak
    \begin{enregistrement}{Racine}
        \champEnregistrement{octet}{\textbf{0..255}}
        \champEnregistrement{frequence}{\naturel}
        \champEnregistrement{arbreGauche}{ArbreDeHuffman}
        \champEnregistrement{arbreDroit}{ArbreDeHuffman}
    \end{enregistrement}
\end{algorithme}

\newpage
\subsection{Algorithmes des fonctions et procédures}

\begin{algorithme}
    \fonction{arbreDeHuffman}{stat : \textbf{Statistiques}, octet : \textbf{0..255}}{\textbf{ArbreDeHuffman}}{}
    {arbre : ArbreDeHuffman}
    {
        \instruction{\allouer{arbre}}
        \affecter{arbre\motclefDereferencer .octet}{stat[octet].octet}
        \affecter{arbre\motclefDereferencer .frequence}{stat[octet].frequence}
        \affecter{arbre\motclefDereferencer .arbreGauche}{\textbf{NIL}}
        \affecter{arbre\motclefDereferencer .arbreDroit}{\textbf{NIL}}
        \retourner{arbre}
    }
\end{algorithme}

\bigbreak
\begin{algorithme}
    \fonction{fusionner}{$a_g$, $a_d$ : \textbf{ArbreDeHuffman}}{\textbf{ArbreDeHuffman}}{}
    {racine : \textbf{ArbreDeHuffman}}
    {
        \instruction{\allouer{racine}}
        \affecter{racine\motclefDereferencer .arbreGauche}{$a_g$}
        \affecter{racine\motclefDereferencer .arbreDroit}{$a_d$}
        \affecter{racine\motclefDereferencer .octet}{0}
        \affecter{racine\motclefDereferencer .frequence}{\textbf{obtenirFrequence}($a_g$) + \textbf{obtenirFrequence}($a_d$)}
        \retourner{racine}
    }
\end{algorithme}

\bigbreak
\begin{algorithme}
    \fonction{estUneFeuille}{arbre : \textbf{ArbreDeHuffman}}{\booleen}{}
    {}
    {
        \retourner{arbre\motclefDereferencer .octet $\ne$ 0}
    }
\end{algorithme}

\bigbreak
\begin{algorithme}
    \fonction{obtenirOctet}{arbre : \textbf{ArbreDeHuffman}}{\textbf{Octet}}{estUneFeuille(arbre)}
    {}
    {
        \retourner{arbre\motclefDereferencer .octet}
    }
\end{algorithme}

\bigbreak
\begin{algorithme}
    \fonction{obtenirFrequence}{arbre : \textbf{ArbreDeHuffman}}{\naturel}{}
    {}
    {
        \retourner{arbre\motclefDereferencer .frequence}
    }
\end{algorithme}

\bigbreak
\begin{algorithme}
    \fonction{obtenirFilsGauche}{arbre : \textbf{ArbreDeHuffman}}{\textbf{ArbreDeHuffman}}{non(estUneFeuille(arbre))}
    {}
    {
        \retourner{arbre\motclefDereferencer .arbreGauche}
    }
\end{algorithme}

\bigbreak
\begin{algorithme}
    \fonction{obtenirFilsDroit}{arbre : \textbf{ArbreDeHuffman}}{\textbf{ArbreDeHuffman}}{non(estUneFeuille(arbre))}
    {}
    {
        \retourner{arbre\motclefDereferencer .arbreDroit}
    }
\end{algorithme}







\end{document}


            \subsection{TAD CodeBinaire}
                \subsection{TAD CodeBinaire}

%\begin{mdframed}
\begin{tad}
  \tadNom{CodeBinaire}
  \tadDependances{\textbf{Octet}, \naturel, \textbf{Bit} }
  \begin{tadOperations}{obtenirLongueur}
  
    \tadOperation{creeCodeBinaire}{bit}{CodeBinaire}
    \tadOperation{ajouterBit}{\tadParams{CodeBinaire, Bit}}{CodeBinaire}
    \tadOperation{retirerBit}{\tadParams{CodeBinaire, Bit}}{CodeBinaire}
    \tadOperationAvecPreconditions{obtenirIemeBit}{\tadParams{CodeBinaire,\naturel}}{Bit}
    \tadOperation{obtenirLongueur}{CodeBinaire}{\naturel}
    \tadOperation{concatener}{\tadParams{CodeBinaire,CodeBinaire}}{CodeBinaire}
     
  \end{tadOperations}


  \begin{tadPreconditions}{ObtenirBit(CodeBinaire)}
    \tadPrecondition{obtenirBit(CodeBinaire,Naturel)}{Non(EstVide(CodeBinaire))}
  \end{tadPreconditions}
    
  
  \begin{tadAxiomes}	
  	    \tadAxiome{obtenirLongueur(concatener(CodeBinaire1,CodeBinaire2))=
  		obtenirLongueur(CodeBinaire1,CodeBinaire2)}
  		\tadAxiome{obtenirLongueur(ajouterBit(CodeBinaire,Bit))=obtenirLongueur(CodeBinaire)+1}
      \end{tadAxiomes}
  
\end{tad}
%\end{mdframed}

            \subsection{TAD TableDeCodage}
                \subsection{TAD Table de Codage}

%\begin{mdframed}
\begin{tad}
    \tadNom{TableDeCodage}
    \tadDependances{\textbf{Octet}, \booleen, \textbf{CodeBinaire}}
    \begin{tadOperations}{octetVersCodeBinaire}
        \tadOperation{creerTableCodage}{\tadParams{\textbf{Octet},\textbf{CodeBinaire}}}{\textbf{TableDeCodage}}
        \tadOperationAvecPreconditions{ajouterCodage}{\tadParams{\textbf{TableDeCodage},\textbf{Octet},\textbf{CodeBinaire}}}{\textbf{TableDeCodage}}
        \tadOperation{codeBinairePresent}{\tadParams{\textbf{TableDeCodage},\textbf{CodeBinaire}}}{\textbf{\booleen}}
        \tadOperation{octetPresent}{\tadParams{\textbf{TableDeCodage},\textbf{Octet}}}{\textbf{\booleen}}
        \tadOperationAvecPreconditions{octetVersCodeBinaire}{\tadParams{\textbf{TableDeCodage},\textbf{Octet}}}{\textbf{CodeBinaire}}
        \tadOperationAvecPreconditions{codeBinaireVersOctet}{\tadParams{\textbf{TableDeCodage},\textbf{CodeBinaire}}}{\textbf{Octet}}
    \end{tadOperations}

    \begin{tadPreconditions}{octetVersCodeBinaire(t, codeBinaire)}
        \tadPrecondition{ajouterCodage(t,octet,codeBinaire)}{non(octetPresent(t, octet)) et non(codeBinairePresent(t, codeBinaire))}
        \tadPrecondition{octetVersCodeBinaire(t, codeBinaire)}{codeBinairePresent(t, codeBinaire)}
        \tadPrecondition{codeBinaireVersOctet(t, octet)}{octetPresent(t,octet)}
    \end{tadPreconditions}

    \begin{tadAxiomes}{}
        octetVersCodeBinaire(ajouterCodage(t, octet, codeBinaire), octet)=codeBinaire\\
        codeBinaireVersOctet(ajouterCodage(t, octet, codeBinaire), codeBinaire)=octet
    \end{tadAxiomes}

\end{tad}
%\end{mdframed}



        \newpage
        \section{Conception Préliminaire}
            \subsection{Signatures et fonctions de Octet}
                \begin{algorithme}
    \signatureFonction{creerOctet}
    {b0, b1, b2, b3, b4, b5, b6, b7 : \textbf{Bit}}
    {\textbf{Octet}}
    {}
    \signatureFonction{obtenirIemeBit}
    {o : \textbf{Octet}, b : \textbf{0..7}}
    {\textbf{Bit}}
    {}
    \signatureFonction{octetVersNaturel}
    {o : \textbf{Octet}}
    {\textbf{0..255}}
    {}
\end{algorithme}

            \subsection{Signatures et fonctions de Statistiques}
                \begin{algorithme}
    \signatureFonction{statistiques}
    {}{\textbf{Statistiques}}
    {}
    \signatureProcedure{incrementerOccurrence}
    {\paramEntreeSortie{s : \textbf{Statistiques}}, \paramEntree{o : \textbf{Octet}}}
    {}
    \signatureFonction{obtenirOccurrence}
    {s : \textbf{Statistiques}, o : \textbf{Octet}}
    {\naturel}
    {}
    \signatureProcedure{fixerOccurrence}
    {\paramEntreeSortie{s : \statistiques}, \paramEntree{o : \octet}, \paramEntree{n : \textbf{Naturel Positif}}}
    {}
\end{algorithme}
            \subsection{Signatures et fonctions de FileDePriorite}
                \subsection{Conception préliminaire : FileDePriorite}

\begin{algorithme}
    \signatureFonction{fileDePriorite}{}{FileDePriorite}{}
    \signatureFonction{estVide}{fdp : FileDePriorite}{Booléen}{}
    \signatureProcedure{enfilerAuBonEndroit}{E/S fdp : FileDePriorite, E e : Element}{}
    \signatureProcedure{obtenirElementEtDefiler}{E/S fdp : FileDePriorite, S Element}{non(estVide(fdp))}
\end{algorithme}
            \subsection{Signatures et fonctions de ArbreDeHuffman}
                \begin{algorithme}

    \signatureFonction{arbreDeHuffman}
    {o : \textbf{Octet}, n : \textbf{Naturel}}
    {\textbf{ArbreDeHuffman}}{}
    \signatureFonction{fusionner}{ag : ArbreDeHuffman, ad : ArbreDeHuffman} {\textbf{ArbreDeHuffman}}{}
    \signatureFonction{estUneFeuille}{a: ArbreDeHuffman}
    {\textbf{Booleen}}{}
    \signatureFonction{obtenirOctect}
    {a: ArbreDeHuffman}
    {\textbf{Octet}}{estUneFeuille(a)}
    \signatureFonction{obtenirFrequence}
    {a: ArbreDeHuffman}
    {\textbf{Naturel}}{}
    \signatureFonction{obtenirFilsGauche}
    {a: ArbreDeHuffman}
    {\textbf{ArbreDeHuffman}}{non(estUneFeuille(a))}
    \signatureFonction{obtenirFilsDroit}
    {a: ArbreDeHuffman}
    {\textbf{ArbreDeHuffman}}{non(estUneFeuille(a))}
    \signatureProcedure{liberer}{\paramEntree{arbre : \arbredehuffman}}{}\commentaire{Procédure métier permettant de libérer un arbre de Huffman de la mémoire}
    
\end{algorithme}

            \subsection{Signatures et fonctions de CodeBinaire}
                \subsection{Conception préliminaire : CodeBinaire}

\begin{algorithme}
    \signatureFonction{creerCodeBinaire}{}{\textbf{CodeBinaire}}{}
    \signatureFonction{estVide}{\textbf{CodeBinaire}}{\textbf{Booleen}}{}
    \signatureProcedure{ajouterBit}{E/S cb : CodeBinaire, E b : bit}{}
    \signatureProcedure{retirerBit}{E/S cb : CodeBinaire, E b : Bit}{non(estVide(cb))}
    \signatureFonction{obtenirIemeBit}{cb : CodeBinaire, i : Naturel}{Bit}{obtenirLongueur(cb)>i}
    \signatureFonction{obtenirLongueur}{cb : CodeBinaire}{Naturel}{} \\
    On ne fait pas les CP et CD de concaténer pour l'instant
\end{algorithme}

            \subsection{Signatures et fonctions de TableDeCodage}
                \subsection{Conception préliminaire : TableDeCodage}

\begin{algorithme}
    \signatureFonction{creerTableCodage}{caractere : Octet , code:CodeBinaire}{TableDeCodage}{}
    \signatureFonction{estVide}{TDC : TableDeCodage}{Booléen}{}
    \signatureFonction{OctetVersCodeBinaire}{TDC : TableDeCodage,caractère :octet}{CodeBinaire}{octetPresent(t,codeBinaire)}
    \signatureFonction{CodeBinaireVersOctet}{TDC : TableDeCodage,code :CodeBinaire}{Octet}{octetPresent(t,octet)}
    \signatureFonction{CodeBinairePresent}{TDC : TableDeCodage,code :CodeBinaire}{Booleen}{}
    \signatureFonction{OctetPresent}{TDC : TableDeCodage,caractère :octet}{Booleen}{}
    \signatureProcedure{ajouterCodage}{E/S TDC : TableDeCodage, E caractere : octet, E code:Codebinaire}{}
    
\end{algorithme}

        \newpage
        \section{Conception Détaillée}
            \textbf{// Dans le but de simplifier la conception détaillée ainsi que le développement, nous considérons bitA0 = 0 et bitA1 = 1. Nous pouvons ainsi utiliser le type Bit comme un naturel. Cela permet d'éviter des instructions conditionnelles répétitives et coûteuses bien que triviales.}
            \subsection{Algorithmes de Octet}
                \subsection{Conception Détaillée : Octet}

\begin{algorithme}
    \type{Octet}{Naturel}
    \\
    \fonction{creerOctet}
    {b0, b1, b2, b3, b4, b5, b6, b7 : \textbf{Bit}}
    {\textbf{Octet}}
    {}
    {o : \textbf{Octet}}
    {  		
        \affecter{o}{b7 + b6*2 + b5*2\^{}2 + b4*2\^{}3 + b3*2\^{}4 + b2*2\^{}5 + b1*2\^{}6 + b0*2\^{}7}
        \retourner{o}
    }
    \\
    \fonction{obtenirIemeBit}
    {o : \textbf{Octet}, b : \textbf{0..7}}
    {\textbf{Bit}}
    {}
    {}
    {
        \retourner{(o \textbf{div} 2\^{}(7-b)) \textbf{mod} 2}
    }
 	\\
    \fonction{octetVersNaturel}
    {o : \textbf{Octet}}
    {\textbf{Naturel}}
    {}
    {}
	{   
    {
       \retourner{o}
    }
    }  
\end{algorithme}

            \subsection{Algorithmes de Statistiques}
                \subsection{Conception Détaillée : Statistiques}

\newcommand{\octet}{\textbf{Octet}}
\newcommand{\codebinaire}{\textbf{CodeBinaire}}
\newcommand{\statistiques}{\textbf{Statistiques}}

\begin{algorithme}
    \type{Statistiques}{Tableau [0..255] de Naturels}
    \\
    \fonction{statistiques}{}{\statistiques}
    {}
    {s : \statistiques}
    {
        \retourner{s}
    }
    \\
    \procedure{incrementerOccurence}{s : \paramEntreeSortie{\statistiques}, o : \paramEntree{\octet}}
    {}
    {}
    {
    	\instruction{s[octetVersNaturel(o)] += 1}
    }
    \\
    \fonction{obtenirOccurence}{s : \statistiques, o : \octet}{\textbf{Naturel}}
    {}
    {}
    {
    	\retourner{s[octetVersNaturel(o)]}
    }
\end{algorithme}
            \subsection{Algorithmes de FileDePriorite}
                \subsection{File de Priorité}
\begin{algorithme}
    \procedure{enfiler}{\paramEntreeSortie{file : \textbf{FileDePriorite}}, \paramEntree{e : \textbf{Element}}}{}
    {temp : \textbf{FileDePriorite}}
    {
         \sialorssinon{estVide(file) ou e $\leq$ file\motclefDereferencer .element}
         {
            \affecter{temp}{file}
            \instruction{\allouer{file}}
            \affecter{file\motclefDereferencer .element}{e}
            \affecter{file\motclefDereferencer .fileSuivante}{temp}
         }
         {
            \instruction{\textbf{enfiler}(file\motclefDereferencer .fileSuivante, e)}{}
         }
    }
\end{algorithme}

\bigbreak
\begin{algorithme}
    \procedure{obtenirElementEtDefiler}{\paramEntreeSortie{file : \textbf{FileDePriorite}, \paramSortie{e : \textbf{Element}}}}{non(estVide(file))}
    {temp : \textbf{FileDePriorite}}
    {
        \affecter{e}{file\motclefDereferencer .element}
        \affecter{temp}{file}
        \affecter{file}{temp\motclefDereferencer .listeSuivante}
        \instruction{\desallouer{temp}}
    }
\end{algorithme}

\bigbreak
\begin{algorithme}
    \fonction{fileDePriorite}{}{\textbf{FileDePriorite}}{}
    {}
    {
        \retourner{\textbf{NIL}}
    }
\end{algorithme}

\bigbreak
\begin{algorithme}
    \fonction{estVide}{file : \textbf{FileDePriorite}}{\booleen}{}
    {}
    {
        \retourner{file = \textbf{NIL}}
    }
\end{algorithme}
            \subsection{Algorithmes de ArbreDeHuffman}
                \begin{algorithme}
	\type{ArbreDeHuffman}{$\widehat{}$ Noeud}
	\begin{enregistrement}{Noeud}
		\champEnregistrement{octet}{Octet}
		\champEnregistrement{frequence}{Naturel}
		\champEnregistrement{estUneFeuille}{Booleen}
		\champEnregistrement{filsGauche}{ArbreDeHuffman}
		\champEnregistrement{filsDroit}{ArbreDeHuffman}
	\end{enregistrement}
	\\
    \fonction{arbreDeHuffman}{o : \textbf{Octet}, f : \naturel}{\textbf{ArbreDeHuffman}}{}
    {a : ArbreDeHuffman}
    {
        \instruction{\allouer{a}}
        \affecter{a\motclefDereferencer .octet}{o}
        \affecter{a\motclefDereferencer .frequence}{f}
        \affecter{a\motclefDereferencer .estUneFeuille}{Vrai}
        \affecter{a\motclefDereferencer .filsGauche}{\textbf{NIL}}
        \affecter{a\motclefDereferencer .filsDroit}{\textbf{NIL}}
        \retourner{a}
    }
	\\
    \fonction{fusionner}{$a_g$, $a_d$ : \textbf{ArbreDeHuffman}}{\textbf{ArbreDeHuffman}}{}
    {racine : \textbf{ArbreDeHuffman}}
    {
        \instruction{\allouer{racine}}
        \affecter{racine\motclefDereferencer .filsGauche}{$a_g$}
        \affecter{racine\motclefDereferencer .filsDroit}{$a_d$}
        \affecter{racine\motclefDereferencer .estUneFeuille}{Faux}
        \affecter{racine\motclefDereferencer .frequence}{\textbf{obtenirFrequence}($a_g$) + \textbf{obtenirFrequence}($a_d$)}
        \retourner{racine}
    }
	\\
    \fonction{estUneFeuille}{a : \textbf{ArbreDeHuffman}}{\booleen}{}
    {}
    {
        \retourner{a\motclefDereferencer .estUneFeuille}
    }
	\\
    \fonction{obtenirOctet}{a : \textbf{ArbreDeHuffman}}{\textbf{Octet}}{estUneFeuille(a)}
    {}
    {
        \retourner{a\motclefDereferencer .octet}
    }
	\\
    \fonction{obtenirFrequence}{a : \textbf{ArbreDeHuffman}}{\naturel}{}
    {}
    {
        \retourner{a\motclefDereferencer .frequence}
    }
	\\
    \fonction{obtenirFilsGauche}{a : \textbf{ArbreDeHuffman}}{\textbf{ArbreDeHuffman}}{non(estUneFeuille(a))}
    {}
    {
        \retourner{a\motclefDereferencer .filsGauche}
    }
	\\
    \fonction{obtenirFilsDroit}{a : \textbf{ArbreDeHuffman}}{\textbf{ArbreDeHuffman}}{non(estUneFeuille(a))}
    {}
    {
        \retourner{a\motclefDereferencer .filsDroit}
    }
    \\
    \procedure{liberer}{\paramEntree{arbre : \arbredehuffman}}{}{}
    {
        \sialors{non(estUneFeuille(arbre))}
        {
            \instruction{liberer(obtenirFilsGauche(arbre))}
            \instruction{liberer(obtenirFilsDroit(arbre))}
        }
        \instruction{desallouer(arbre)}
    }
\end{algorithme}

            \subsection{Algorithmes de CodeBinaire}
                \begin{algorithme}
    \begin{enregistrement}{CodeBinaire}
        \champEnregistrement{tableBit}{\tableauUneDimension{0..7}{de }{\textbf{Bit}}}
        \champEnregistrement{nbElements}{\textbf{Naturel}}
    \end{enregistrement}
    \\
    \fonction{creeCodeBinaire}
    {b: \textbf{Bit}}
    {\textbf{CodeBinaire}}
    {}
    {cb : \textbf{CodeBinaire}}
    {  		
    	\affecter{cb.tablebit[0]}{b}
    	\affecter{cb.nbElements}{1}
        \retourner{cb}
    }
    \\
    \fonction{obtenirIemeBit}
    {cb : \textbf{CodeBinaire}, i: \textbf{Naturel}}
    {\textbf{Bit}}
    {i<=obtenirLongueur(cb)}
    {}
    {
        \retourner{cb.tableBit[i]}
    }
 	\\
    \fonction{obtenirLongueur}{cb : CodeBinaire}
    {\textbf{Naturel}}
    {}
    {}
    {
       \retourner{cb.nbElements}
    }
    \\
     \procedure{ajouterBit}{E/S cb : CodeBinaire, E b : bit}
    {}
    {}
    {
       \affecter {cb.tableBit[obtenirLongueur(cd)+1]}{b}
	    \affecter {cb.nbElements}{obtenirLongueur(cd)+1}
    }
\end{algorithme}

            \subsection{Algorithmes de TableDeCodage}
                \section{Conception Détaillée : TableDeCodage}

\newcommand{\octet}{\textbf{Octet}}
\newcommand{\codebinaire}{\textbf{CodeBinaire}}
\newcommand{\tabledecodage}{\textbf{TableDeCodage}}

\begin{algorithme}
    \type{TableDeCodage}{Dictionnaire<Octet,CodeBinaire>}
    \\
    \fonction{creerTableCodage}{}{\tabledecodage}
    {}
    {}
    {
        \retourner{dictionnaire()}
    }
    \\
    \procedure{ajouterCodage}{tdc : \tabledecodage, o : \octet, cb : \codebinaire}
    {non(octetPresent(t, octet))}
    {}
    {
        \instruction{ajouter(tdc,o,cb)}
    }
    \\
    \fonction{octetVersCodeBinaire}{tdc : \tabledecodage, o : \octet}{\codebinaire}
    {octetPresent(t, octet)}
    {}
    {
        \retourner{obtenirValeur(tdc,o)}
    }
    \\
    \fonction{octetPresent}{tdc : \tabledecodage, o : \octet}{\booleen}
    {}
    {}
    {
        \retourner{estPresent(tdc,o)}
    }
\end{algorithme}
