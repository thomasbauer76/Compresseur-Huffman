\begin{algorithme}
    \begin{enregistrement}{CodeBinaire}
        \champEnregistrement{codeBinaire}{\textbf{Naturel}}
        \champEnregistrement{nbBits}{\textbf{Naturel}}
    \end{enregistrement}
    \\
    \fonction{creerCodeBinaire}
    {b: \textbf{Bit}}
    {\textbf{CodeBinaire}}
    {}
    {cb : \textbf{CodeBinaire}}
    {  		
        \sialorssinon{b = bitA1}
        {
            \affecter{cb.codeBinaire}{1}
        }
        {
            \affecter{cb.codeBinaire}{0}
        }
    	\affecter{cb.nbBits}{1}
        \retourner{cb}
    }
    \\
    \fonction{obtenirIemeBit}
    {cb : \textbf{CodeBinaire}, i: \textbf{0..7}}
    {\textbf{Bit}}
    {}
    {}
    {
        \retourner{(cb.codeBinaire \textbf{div} 2\^{}i) \textbf{mod} 2}
    }
 	\\
    \fonction{obtenirLongueur}{cb : CodeBinaire}
    {\textbf{Naturel}}
    {}
    {}
    {
       \retourner{cb.nbBits}
    }
    \\
     \procedure{ajouterBit}{E/S cb : CodeBinaire, E b : bit}
    {}
    {}
    {   
        \sialors{b = bitA1}
        {
            \affecter{cb.codebinaire}{cb.codebinaire + 2\^{}(cb.nbBits)}
        }
	    \affecter{cb.nbBits}{cb.nbBits + 1}
    }
\end{algorithme}
