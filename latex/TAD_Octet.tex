\documentclass[10pt]{article}
\usepackage[utf8]{inputenc} % encodage du fichier
\usepackage[french]{babel} % document en francais
\usepackage[T1]{fontenc} % pour les caracteres accentues
\usepackage{geometry}
\geometry{a4paper, left=2.5cm, right=2.5cm, top=2.5cm, bottom=2.5cm}
\usepackage{pseudocode}
\usepackage{mdframed}

\begin{document}

\section{TAD Table de Codage}

%\begin{mdframed}
\begin{tad}
    \tadNom{Octet}
    \tadDependances{\textbf{Bit}, \textbf{[1..8]}}
    \begin{tadOperations}{octetVersNaturel}
        \tadOperation{creerOctet}{\tadParams{\textbf{Bit},\textbf{Bit},\textbf{Bit},\textbf{Bit},\textbf{Bit},\textbf{Bit},\textbf{Bit},\textbf{Bit}}}{\textbf{Octet}}
        \tadOperation{obtenirIemeBit}{\tadParams{\textbf{Octet},\textbf{[1..8]}}}{\textbf{Bit}}
        \tadOperation{octetVersNaturel}{\textbf{Octet}}{\textbf{[1..256]}}
        \tadOperation{naturelVersOctet}{\textbf{[1..256]}}{\textbf{Octet}}
       
       
    \end{tadOperations}

    \begin{tadPreconditions}{octetVersCodeBinaire(t, codeBinaire)}
        \tadPrecondition{Aucune}
    \end{tadPreconditions}

    \begin{tadAxiomes}{}
        A mieux faire 	après avoir revu TOUTES les opérations du TAD
    \end{tadAxiomes}

\end{tad}
%\end{mdframed}




\end{document}

