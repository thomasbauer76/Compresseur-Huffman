\subsection{TAD Octet}

%\begin{mdframed}
\begin{tad}
    \tadNom{Octet}
    \tadDependances{\textbf{Bit}, \textbf{0..7}}
    \begin{tadOperations}{octetVersNaturel}
        \tadOperation{creerOctet}{\tadParams{\textbf{Bit},\textbf{Bit},\textbf{Bit},\textbf{Bit},\textbf{Bit},\textbf{Bit},\textbf{Bit},\textbf{Bit}}}{\textbf{Octet}}
        \tadOperation{obtenirIemeBit}{\tadParams{\textbf{Octet},\textbf{0..7}}}{\textbf{Bit}}
        \tadOperation{octetVersNaturel}{\textbf{Octet}}{\textbf{0..255}}    
    \end{tadOperations}

    \begin{tadAxiomes}
        \tadAxiome{obtenirIemeBit(creerOctet(b_0,b_1,b_2,b_3,b_4,b_5,b_6,b_7),0)= b_0}\\
        \textit{Note : on obtient $b_1$ pour l'indice 1, ... et $b_7$ pour l'indice 7}
    \end{tadAxiomes}
    
    \begin{tadSemantiques}{octetVersNaturel}
    	\tadSemantique{octetVersNaturel}{permet la conversion d'un nombre en base 2 (Octet) vers un nombre en base 10 (Naturel)}
    \end{tadSemantiques}

\end{tad}
%\end{mdframed}
