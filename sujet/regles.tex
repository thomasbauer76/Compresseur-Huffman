\documentclass{article}
\usepackage{graphicx}

\begin{document}

\title{Règles de Programmation en C}

\date{\today}

\maketitle

\section*{Conventions code}
\subsection{Lisibilité du Code}
\begin{enumerate}
    \item \textbf{Règle 1 —} Ouvrir une fonction en bout de ligne et la fermer en fin dans la meme ligne. Utilisation systématique des accolades pour les conditionnelles
et les boucles.
    \item \textbf{Règle 2 —} Espaces avant et après chaque opérateur.
    \item \textbf{Règle 3 —} Espace après chaque virgule.
    \item \textbf{Règle 4 —} Une tabulation pour indentation.
    \item \textbf{Règle 5 —} Chaque opération booléenne ou simple sera entre parenthèses.
    \item \textbf{Règle 6 —} Éviter les chiffres sans définitions.
    
     \item \textbf{Règle 8 —} Seuls les types \textit{signed char} et \textit{unsigned char} 
doivent être utilisés pour manipuler des valeurs numériques.
     \item \textbf{Règle 9 —}  Indentation des expressions longues.
\subsection{Regles et bonnes pratiques.}  
    \item \textbf{Règle 10 —} Envoi de code qui compile sans erreurs ni avertissements.
    \item \textbf{Règle 11 —} Optimiser les codes.
    \item \textbf{Règle 12 —} Ne pas faire de déclaration multiple de variables associée à une initialisation.
    \item \textbf{Règle 13 —} Limiter les variables globales au strict nécessaire.
    \item \textbf{Règle 14 —} Centraliser les constantes en debut de programme et les ecrire en majuscule.
     \item \textbf{Règle 15 —} Toutes les variables utilisées doivent avoir été préalablement déclarées de façon ex-plicite.
     \item \textbf{Règle 16 —} Toutes les variables doivent être systématiquement initialisées à leur déclaration ou
immédiatement après dans le cas de déclarations multiples.
	 \item \textbf{Règle 17 —} Toutes les variables déclarée doivent être utilisée.
    \item \textbf{Règle 18 —} Retour implicite interdit pour les fonctions de type non \texttt{void}
    \item \textbf{Règle 19 —} Allouer dynamiquement un espace mémoire dont la taille est suffisante pour l’objet alloué
    \item \textbf{Règle 20 —} Libérer la mémoire allouée dynamiquement au plus tôt
    \item \textbf{Règle 21 —} Vérification obligatoire du succès d’une allocation mémoire
    \item \textbf{Règle 22 —} Un pointeur doit être affecté à NULL après désallocation

	\item \textbf{Règle 23 —} Préférer l'utilisation de l'opérateur d'indirection ->.
	\item \textbf{Règle 24 —} Une seule instruction par ligne de code.
	\item \textbf{Règle 25 —} Eviter l'utilisation des nombres flottants au maximum.
	\item \textbf{Règle 26 —} La valeur de retour d'une fonction doit toujours être testée.
	\item \textbf{Règle 27 —} Usage de la virgule interdit pour le séquencement d'instructions.
	\item \textbf{Règle 28 —} Privilégier les retours d'erreurs via des codes de
retour dans la fonction principale.
	\item \textbf{Règle 29 —} Tout fichier non vide doit se terminer par un retour à la ligne et
les directives de préprocesseur et les commentaires doivent être fermés.
	\item \textbf{Règle 30 —} Préférer l'utilisation de l'opérateur d'indirection ->


 
\end{enumerate}

\begin{figure}[h]
    \centering
\includegraphics[width=1.5\textwidth]{../../Images/Captures d’écran/Capture d’écran du 2023-12-16 20-46-54.png}     
    \caption{Exemple de Code.}
    \label{fig:votre_image}
\end{figure}


\begin{itemize}
    \item \textbf{Source }ANSSI Guide Règles de Programmation pour le Développement Sécurisé de Logiciels en Langage C 
   \end{itemize}
\end{document}

