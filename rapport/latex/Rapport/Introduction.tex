\section{Introdution}
	La compression de données est un domaine de recherche actif depuis de nombreuses années. Elle vise à réduire la taille des données en conservant le contenu original. La compression est utilisée dans de nombreuses applications, telles que le stockage de données, la transmission de données et le traitement des données.
	\bigbreak
	Parmi les nombreux paradigmes de compression, les deux catégories prédominantes sont les algorithmes de compression par perte et sans perte. Les premiers, sacrifiant une partie de l'information originale pour une compression plus importante, sont souvent utilisés dans des applications où une légère dégradation de la qualité est acceptable. En revanche, les algorithmes sans perte préservent intégralement les données initiales, trouvant leur utilité dans des domaines sensibles à toute altération, tels que les archives numériques, les bases de données et les transmissions sans erreur.
	\bigbreak
	Le présent projet se fixe pour objectif de concevoir un algorithme de compression sans perte, s'appuyant sur la technique du codage de Huffman. Une fois l'algorithme de compression développé, une évaluation de son efficacité sera réalisée par le biais de tests de compression de fichiers de différentes tailles, types et contenus.
	\bigbreak
	La structure du rapport de ce projet suivra un plan comprenant la présentation des Types Abstraits de Données (TADs) et des analyses descendantes, la conception préliminaire, la conception détaillée, l'implémentation du code C et des tests unitaires ainsi qu'une section dédiée à l'organisation du groupe. Enfin, le rapport se conclura par une synthèse globale du projet et des retours sur les résultats obtenus.
