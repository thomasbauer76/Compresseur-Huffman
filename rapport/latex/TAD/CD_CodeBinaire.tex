\begin{algorithme}
    \begin{enregistrement}{CodeBinaire}
        \champEnregistrement{codeBinaire}{\textbf{Naturel}}
        \champEnregistrement{nbBits}{\textbf{Naturel}}
    \end{enregistrement}
    \\
    \fonction{creerCodeBinaire}
    {b: \bit}
    {\codebinaire}
    {}
    {cb : \codebinaire}
    {  		
        \affecter{cb.codeBinaire}{b}
    	\affecter{cb.nbBits}{1}
        \retourner{cb}
    }
    \\
    \fonction{obtenirIemeBit}
    {cb : \codebinaire, i : \naturel}
    {\bit}
    {i < cb.nbBits}
    {}
    {
        \retourner{(cb.codeBinaire \textbf{div} 2\^{}i) \textbf{mod} 2}
    }
 	\\
    \fonction{obtenirLongueur}{cb : \codebinaire}
    {\textbf{Naturel}}
    {}
    {}
    {
       \retourner{cb.nbBits}
    }
    \\
    \procedure{ajouterBit}{\paramEntreeSortie{cb : \codebinaire}, \paramEntree{b : bit}}
    {}
    {}
    {   
        \affecter{cb.codebinaire}{cb.codebinaire + b * 2\^{}(cb.nbBits)}
	    \affecter{cb.nbBits}{cb.nbBits + 1}
    }
\end{algorithme}
